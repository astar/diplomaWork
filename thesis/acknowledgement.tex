% Thesis Acknowledgements ------------------------------------------------


%\begin{acknowledgementslong} %uncommenting this line, gives a different acknowledgements heading
\begin{acknowledgements}      %this creates the heading for the acknowledgments


  I would like to thank Filip Hroch and Petr Škoda for their
  remarkable support and patience not only during this project. I
  greatly appreciated comments and help from following friends: Tereza
  Jeřábková, Josef Pacula, Petr Šafařík, a vsem tem ubozakum co je to
  donutim precist;-) 


  \bigskip 
  Funding for the SDSS and SDSS-II has been provided by the
  Alfred P. Sloan Foundation, the Participating Institutions, the
  National Science Foundation, the U.S. Department of Energy, the
  National Aeronautics and Space Administration, the Japanese
  Monbukagakusho, the Max Planck Society, and the Higher Education
  Funding Council for England. The SDSS Web Site is
  \url{http://www.sdss.org/}.

  The SDSS is managed by the Astrophysical Research Consortium for the
  Participating Institutions. The Participating Institutions are the
  American Museum of Natural History, Astrophysical Institute Potsdam,
  University of Basel, University of Cambridge, Case Western Reserve
  University, University of Chicago, Drexel University, Fermilab, the
  Institute for Advanced Study, the Japan Participation Group, Johns
  Hopkins University, the Joint Institute for Nuclear Astrophysics,
  the Kavli Institute for Particle Astrophysics and Cosmology, the
  Korean Scientist Group, the Chinese Academy of Sciences (LAMOST),
  Los Alamos National Laboratory, the Max-Planck-Institute for
  Astronomy (MPIA), the Max-Planck-Institute for Astrophysics (MPA),
  New Mexico State University, Ohio State University, University of
  Pittsburgh, University of Portsmouth, Princeton University, the
  United States Naval Observatory, and the University of Washington.

  \bigskip

  This research has made use of the SIMBAD database, operated at CDS,
  Strasbourg, France.

  \bigskip

  This research has made use of The WEKA Data Mining Software
  \citep{hall2009weka}.




\end{acknowledgements}
%\end{acknowledgmentslong}

% ------------------------------------------------------------------------

%%% Local Variables: 
%%% mode: latex
%%% TeX-master: "../thesis"
%%% End: 
