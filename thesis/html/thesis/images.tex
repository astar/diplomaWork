\batchmode


 \documentclass[oneside,12pt]{Classes/CUEDthesisPSnPDF}
 \usepackage[margin=2.5cm]{geometry}


\ifpdf
    \pdfinfo { /Title  (Thesis)
               /Creator (TeX)
               /Producer (pdfTeX)
               /Author (Jaroslav Vazny jaroslav.vazny@gmail.com)
               /CreationDate (D:20030101000000)  %format D:YYYYMMDDhhmmss
               /ModDate (D:20030815213532)
               /Subject (Thesis)
               /Keywords (Master, Thesis)}
    \pdfcatalog { /PageMode (/UseOutlines)
                  /OpenAction (fitbh)  }
\fi


\title{Virtual Observatory}


\ifpdf
  \author{\href{mailto:jaroslav.vazny@gmail.com}{Jaroslav Vazny}}
  \collegeordept{\href{http://www.physics.muni.cz}{Department of Theoretical Physics and Astrophysics }}
  \university{\href{http://www.muni.cz}{Masarykova Univerzita}}
\crest{\includegraphics[width=60mm]{ivoa_logo}}
\else
  \author{Jaroslav Vazny}
  \collegeordept{Department of Theoretical Physics and Astrophysics}
  \university{Masarykova univerzita}
\crest{\includegraphics[bb = 0 0 292 336, width=30mm]{pic/ivoa_logo.png}}
\fi
\degree{Master}
\degreedate{Yet to be decided}


\hbadness=10000
\hfuzz=50pt


\usepackage{StyleFiles/watermark}
\usepackage{type1cm}
\usepackage[Lenny]{StyleFiles/fncychapleo}
%
\renewcommand{\chaptermark}[1]{\markboth{\MakeUppercase{#1}}{}}
%
\renewcommand{\sectionmark}[1]{\markright{\MakeUppercase{#1}}{}}
%
\renewcommand{\headrulewidth}{0pt}


\onehalfspacing




\usepackage[dvips]{color}


\pagecolor[gray]{.7}

\usepackage[]{inputenc}



\makeatletter

\makeatletter
\count@=\the\catcode`\_ \catcode`\_=8 
\newenvironment{tex2html_wrap}{}{}%
\catcode`\<=12\catcode`\_=\count@
\newcommand{\providedcommand}[1]{\expandafter\providecommand\csname #1\endcsname}%
\newcommand{\renewedcommand}[1]{\expandafter\providecommand\csname #1\endcsname{}%
  \expandafter\renewcommand\csname #1\endcsname}%
\newcommand{\newedenvironment}[1]{\newenvironment{#1}{}{}\renewenvironment{#1}}%
\let\newedcommand\renewedcommand
\let\renewedenvironment\newedenvironment
\makeatother
\let\mathon=$
\let\mathoff=$
\ifx\AtBeginDocument\undefined \newcommand{\AtBeginDocument}[1]{}\fi
\newbox\sizebox
\setlength{\hoffset}{0pt}\setlength{\voffset}{0pt}
\addtolength{\textheight}{\footskip}\setlength{\footskip}{0pt}
\addtolength{\textheight}{\topmargin}\setlength{\topmargin}{0pt}
\addtolength{\textheight}{\headheight}\setlength{\headheight}{0pt}
\addtolength{\textheight}{\headsep}\setlength{\headsep}{0pt}
\setlength{\textwidth}{349pt}
\newwrite\lthtmlwrite
\makeatletter
\let\realnormalsize=\normalsize
\global\topskip=2sp
\def\preveqno{}\let\real@float=\@float \let\realend@float=\end@float
\def\@float{\let\@savefreelist\@freelist\real@float}
\def\liih@math{\ifmmode$\else\bad@math\fi}
\def\end@float{\realend@float\global\let\@freelist\@savefreelist}
\let\real@dbflt=\@dbflt \let\end@dblfloat=\end@float
\let\@largefloatcheck=\relax
\let\if@boxedmulticols=\iftrue
\def\@dbflt{\let\@savefreelist\@freelist\real@dbflt}
\def\adjustnormalsize{\def\normalsize{\mathsurround=0pt \realnormalsize
 \parindent=0pt\abovedisplayskip=0pt\belowdisplayskip=0pt}%
 \def\phantompar{\csname par\endcsname}\normalsize}%
\def\lthtmltypeout#1{{\let\protect\string \immediate\write\lthtmlwrite{#1}}}%
\newcommand\lthtmlhboxmathA{\adjustnormalsize\setbox\sizebox=\hbox\bgroup\kern.05em }%
\newcommand\lthtmlhboxmathB{\adjustnormalsize\setbox\sizebox=\hbox to\hsize\bgroup\hfill }%
\newcommand\lthtmlvboxmathA{\adjustnormalsize\setbox\sizebox=\vbox\bgroup %
 \let\ifinner=\iffalse \let\)\liih@math }%
\newcommand\lthtmlboxmathZ{\@next\next\@currlist{}{\def\next{\voidb@x}}%
 \expandafter\box\next\egroup}%
\newcommand\lthtmlmathtype[1]{\gdef\lthtmlmathenv{#1}}%
\newcommand\lthtmllogmath{\dimen0\ht\sizebox \advance\dimen0\dp\sizebox
  \ifdim\dimen0>.95\vsize
   \lthtmltypeout{%
*** image for \lthtmlmathenv\space is too tall at \the\dimen0, reducing to .95 vsize ***}%
   \ht\sizebox.95\vsize \dp\sizebox\z@ \fi
  \lthtmltypeout{l2hSize %
:\lthtmlmathenv:\the\ht\sizebox::\the\dp\sizebox::\the\wd\sizebox.\preveqno}}%
\newcommand\lthtmlfigureA[1]{\let\@savefreelist\@freelist
       \lthtmlmathtype{#1}\lthtmlvboxmathA}%
\newcommand\lthtmlpictureA{\bgroup\catcode`\_=8 \lthtmlpictureB}%
\newcommand\lthtmlpictureB[1]{\lthtmlmathtype{#1}\egroup
       \let\@savefreelist\@freelist \lthtmlhboxmathB}%
\newcommand\lthtmlpictureZ[1]{\hfill\lthtmlfigureZ}%
\newcommand\lthtmlfigureZ{\lthtmlboxmathZ\lthtmllogmath\copy\sizebox
       \global\let\@freelist\@savefreelist}%
\newcommand\lthtmldisplayA{\bgroup\catcode`\_=8 \lthtmldisplayAi}%
\newcommand\lthtmldisplayAi[1]{\lthtmlmathtype{#1}\egroup\lthtmlvboxmathA}%
\newcommand\lthtmldisplayB[1]{\edef\preveqno{(\theequation)}%
  \lthtmldisplayA{#1}\let\@eqnnum\relax}%
\newcommand\lthtmldisplayZ{\lthtmlboxmathZ\lthtmllogmath\lthtmlsetmath}%
\newcommand\lthtmlinlinemathA{\bgroup\catcode`\_=8 \lthtmlinlinemathB}
\newcommand\lthtmlinlinemathB[1]{\lthtmlmathtype{#1}\egroup\lthtmlhboxmathA
  \vrule height1.5ex width0pt }%
\newcommand\lthtmlinlineA{\bgroup\catcode`\_=8 \lthtmlinlineB}%
\newcommand\lthtmlinlineB[1]{\lthtmlmathtype{#1}\egroup\lthtmlhboxmathA}%
\newcommand\lthtmlinlineZ{\egroup\expandafter\ifdim\dp\sizebox>0pt %
  \expandafter\centerinlinemath\fi\lthtmllogmath\lthtmlsetinline}
\newcommand\lthtmlinlinemathZ{\egroup\expandafter\ifdim\dp\sizebox>0pt %
  \expandafter\centerinlinemath\fi\lthtmllogmath\lthtmlsetmath}
\newcommand\lthtmlindisplaymathZ{\egroup %
  \centerinlinemath\lthtmllogmath\lthtmlsetmath}
\def\lthtmlsetinline{\hbox{\vrule width.1em \vtop{\vbox{%
  \kern.1em\copy\sizebox}\ifdim\dp\sizebox>0pt\kern.1em\else\kern.3pt\fi
  \ifdim\hsize>\wd\sizebox \hrule depth1pt\fi}}}
\def\lthtmlsetmath{\hbox{\vrule width.1em\kern-.05em\vtop{\vbox{%
  \kern.1em\kern0.8 pt\hbox{\hglue.17em\copy\sizebox\hglue0.8 pt}}\kern.3pt%
  \ifdim\dp\sizebox>0pt\kern.1em\fi \kern0.8 pt%
  \ifdim\hsize>\wd\sizebox \hrule depth1pt\fi}}}
\def\centerinlinemath{%
  \dimen1=\ifdim\ht\sizebox<\dp\sizebox \dp\sizebox\else\ht\sizebox\fi
  \advance\dimen1by.5pt \vrule width0pt height\dimen1 depth\dimen1 
 \dp\sizebox=\dimen1\ht\sizebox=\dimen1\relax}

\def\lthtmlcheckvsize{\ifdim\ht\sizebox<\vsize 
  \ifdim\wd\sizebox<\hsize\expandafter\hfill\fi \expandafter\vfill
  \else\expandafter\vss\fi}%
\providecommand{\selectlanguage}[1]{}%
\makeatletter \tracingstats = 1 


\begin{document}
\pagestyle{empty}\thispagestyle{empty}\lthtmltypeout{}%
\lthtmltypeout{latex2htmlLength hsize=\the\hsize}\lthtmltypeout{}%
\lthtmltypeout{latex2htmlLength vsize=\the\vsize}\lthtmltypeout{}%
\lthtmltypeout{latex2htmlLength hoffset=\the\hoffset}\lthtmltypeout{}%
\lthtmltypeout{latex2htmlLength voffset=\the\voffset}\lthtmltypeout{}%
\lthtmltypeout{latex2htmlLength topmargin=\the\topmargin}\lthtmltypeout{}%
\lthtmltypeout{latex2htmlLength topskip=\the\topskip}\lthtmltypeout{}%
\lthtmltypeout{latex2htmlLength headheight=\the\headheight}\lthtmltypeout{}%
\lthtmltypeout{latex2htmlLength headsep=\the\headsep}\lthtmltypeout{}%
\lthtmltypeout{latex2htmlLength parskip=\the\parskip}\lthtmltypeout{}%
\lthtmltypeout{latex2htmlLength oddsidemargin=\the\oddsidemargin}\lthtmltypeout{}%
\makeatletter
\if@twoside\lthtmltypeout{latex2htmlLength evensidemargin=\the\evensidemargin}%
\else\lthtmltypeout{latex2htmlLength evensidemargin=\the\oddsidemargin}\fi%
\lthtmltypeout{}%
\makeatother
\setcounter{page}{1}
\onecolumn

% !!! IMAGES START HERE !!!

\setcounter{secnumdepth}{3}
\setcounter{tocdepth}{3}
{\newpage\clearpage
\lthtmlfigureA{dedication109}%
\begin{dedication} %this creates the heading for the dedication page
\par
I would like to dedicate this thesis to my loving parents ...
\par
\end{dedication}%
\lthtmlfigureZ
\lthtmlcheckvsize\clearpage}

{\newpage\clearpage
\lthtmlfigureA{acknowledgements119}%
\begin{acknowledgements}      %this creates the heading for the acknowlegments
\par
And I would like to acknowledge ...
\par
\end{acknowledgements}%
\lthtmlfigureZ
\lthtmlcheckvsize\clearpage}

{\newpage\clearpage
\lthtmlfigureA{abstracts129}%
\begin{abstracts}        %this creates the heading for the abstract page
\par
This is where you write your abstract ...
\par
\end{abstracts}%
\lthtmlfigureZ
\lthtmlcheckvsize\clearpage}


\renewcommand{\LettrineFontHook}{\color{red}}
\stepcounter{chapter}
{\newpage\clearpage
\lthtmlfigureA{lstlisting198}%
\begin{lstlisting}[frame=single]
What is the motivation behind Virtual Observatory? Is data avalanche
problem only in astronomy? What is IVOA?  What is Virtual Observatory
architecture?
\end{lstlisting}%
\lthtmlfigureZ
\lthtmlcheckvsize\clearpage}

\stepcounter{section}
\stepcounter{section}
{\newpage\clearpage
\lthtmlfigureA{wrapfigure210}%
\begin{wrapfigure}
% latex2html id marker 210
{r}{0.5\textwidth}
     \vspace{0pt}
     \begin{center}
       \ifpdf
       \includegraphics[width=0.4\textwidth]{ivoamembers}
       \else
       \includegraphics[bb = 92 86 545 742, height=6in]{ivoamembers.jpg}
       \fi
     \end{center}
     \vspace{-20pt}
     \caption{IVOA members}
     \vspace{-10pt}
   \end{wrapfigure}%
\lthtmlfigureZ
\lthtmlcheckvsize\clearpage}

\stepcounter{section}
\stepcounter{section}
{\newpage\clearpage
\lthtmlfigureA{lstlisting250}%
\begin{lstlisting}[frame=single]
stilts regquery query="shortName like 'AIASCR'"
regurl=http://registry.euro-vo.org/services/RegistrySearch
ofmt=votable-tabledata > resourceExample.vot
\end{lstlisting}%
\lthtmlfigureZ
\lthtmlcheckvsize\clearpage}

{\newpage\clearpage
\lthtmlfigureA{lstlisting330}%
\begin{lstlisting}[frame=single]
<?xml version='1.0'?>
<VOTABLE version="1.1"
 xmlns:xsi="http://www.w3.org/2001/XMLSchema-instance"
 xsi:schemaLocation="http://www.ivoa.net/xml/VOTable/v1.1 http://www.ivoa.net/xml/VOTable/v1.1"
 xmlns="http://www.ivoa.net/xml/VOTable/v1.1">
<!--
 !  VOTable written by STIL version 3.0 (uk.ac.starlink.votable.VOTableWriter)
 !  at 2011-03-24T00:45:59
 !-->
<RESOURCE>
<TABLE nrows="1">
<LINK title="Registry Location" href="http://registry.euro-vo.org/services/RegistrySearch"/>
<PARAM arraysize="23" datatype="char" name="Registry Query" value="shortName like 'AIASCR'">
<DESCRIPTION>Text of query made to the registry</DESCRIPTION>
</PARAM>
.
.
.
<DATA>
<TABLEDATA>
  <TR>
    <TD>ivo://asu.cas.cz</TD>
    <TD>AIASCR</TD>
    <TD>Astronomical Institute of the Academy of Sciences of the Czech Republic Naming Authority</TD>
    <TD>Astronomical Institute of the Academy of Sciences of the Czech Republic</TD>
    <TD>http://stelweb.asu.cas.cz/web/index/index-en.php</TD>
    <TD>Petr Skoda \ensuremath{<}skoda@sunstel.asu.cas.cz\ensuremath{>}</TD>
  </TR>
</TABLEDATA>
</DATA>
</TABLE>
</RESOURCE>
</VOTABLE>
\end{lstlisting}%
\lthtmlfigureZ
\lthtmlcheckvsize\clearpage}

\stepcounter{section}
\stepcounter{subsection}
\stepcounter{subsection}
\stepcounter{subsection}
\stepcounter{section}
\stepcounter{subsection}
{\newpage\clearpage
\lthtmlfigureA{lstlisting343}%
\begin{lstlisting}
<?xml version="1.0" encoding="utf-8"?>
<!-- Produced with vo.table version 0.6
     http://www.stsci.edu/trac/ssb/astrolib
     Author: Michael Droettboom <support@stsci.edu> -->
<VOTABLE version="1.0"
 xmlns:xsi="http://www.w3.org/2001/XMLSchema-instance"
 xsi:noNamespaceSchemaLocation="http://www.ivoa.net/xml/VOTable/v1.0"
 xmlns="http://www.ivoa.net/xml/VOTable/v1.0">
 <RESOURCE type="results" >
  <TABLE >
   <FIELD ID="col0" name="wave" datatype="float" unit=""
   precision="F9"/>
  <DATA>
    <TABLEDATA>
     <TR>
      <TD>4012.50757</TD>
     </TR>
 </TABLEDATA>
   </DATA>
  </TABLE>
 </RESOURCE>
</VOTABLE>
\end{lstlisting}%
\lthtmlfigureZ
\lthtmlcheckvsize\clearpage}

{\newpage\clearpage
\lthtmlfigureA{lstlisting348}%
\begin{lstlisting}
  In [1]: import atpy
  In [2]: tbl = atpy.Table('spSpec-53401-2052-458.fit',hdu=1)
  Auto-detected input type: fits
  In [3]: tbl.write('votableExample.xml')
  Auto-detected input type: vo
\end{lstlisting}%
\lthtmlfigureZ
\lthtmlcheckvsize\clearpage}

\stepcounter{subsection}
{\newpage\clearpage
\lthtmlfigureA{lstlisting364}%
\begin{lstlisting}
In [1]: import pyfits
In [2]: hdulist = pyfits.open('spSpec-53237-1886-248.fit')
In [3]: hdulist.info()
Filename: spSpec-53237-1886-248.fit
No.    Name         Type      Cards   Dimensions   Format
0    PRIMARY     PrimaryHDU     213  (3874, 5)     float32
1                BinTableHDU     54  6R x 23C      [1E, 1E, ...
2                BinTableHDU     54  44R x 23C     [1E, 1E, ...
3                BinTableHDU     18  1R x 5C       [1E, 1E, ...
4                BinTableHDU     32  53R x 12C     [1J, 1J, ...
5                BinTableHDU     26  36R x 9C      [19A, 1E, ...
6                BinTableHDU     14  3874R x 3C    [1J, 1J, 1E]
\end{lstlisting}%
\lthtmlfigureZ
\lthtmlcheckvsize\clearpage}

{\newpage\clearpage
\lthtmlfigureA{lstlisting366}%
\begin{lstlisting}
In [4]: print hdulist[0].header
-------> print(hdulist[0].header)
DATE-OBS= '2004-08-20'         / 1st row - TAI date                             
TAIHMS  = '10:36:18.11'        / 1st row - TAI time (HH:MM:SS.SS) (TAI-UT = appr
TIMESYS = 'tai     '           / TAI, not UTC                                   
TAI-BEG =        4599713999.00 / Exposure Start Time                            
TAI-END =        4599717089.00 / Exposure End Time                              
MJD     =                53237 / MJD of observation                             
MJDLIST = '53237   '           /                                                
VERSION = 'v3_140_0'           / version of IOP                                 
CAMVER  = 'SPEC1 v4_8'         / Camera code version                            
OBSERVER= 'prn     '                                                            
OBSCOMM = 'science '                                                            
TELESCOP= 'SDSS 2.5-M'         / Sloan Digital Sky Survey
\end{lstlisting}%
\lthtmlfigureZ
\lthtmlcheckvsize\clearpage}

{\newpage\clearpage
\lthtmlfigureA{lstlisting368}%
\begin{lstlisting}
  In [16]: prihdr = hdulist[0].header
  In [17]: prihdr.update('observer', 'Astar')
  In [18]: prihdr.add_history('I updated this file 3/27/11')
\end{lstlisting}%
\lthtmlfigureZ
\lthtmlcheckvsize\clearpage}

{\newpage\clearpage
\lthtmlinlinemathA{tex2html_wrap_inline380}%
$H\alpha$%
\lthtmlinlinemathZ
\lthtmlcheckvsize\clearpage}

{\newpage\clearpage
\lthtmlfigureA{lstlisting370}%
\begin{lstlisting}
 def read(file):
    """ Read fits file. Convert wavelength to angstroms """ 
    data = pyfits.getdata(file)
    w = lambda x : 10.0**(3.5796 + x*10.0**(-4))
    x = np.arange(1,data[0].size + 1)
    xx  = w(x) # convert to actual wavelenght
    return np.asarray([xx, data[0]])
\par
def plot(file,xdata,ydata,spLine):
    fig = plt.figure()
    ax = fig.add_subplot(111)
    graph = ax.plot(xdata,ydata, 'r')
    ax.set_title(file)
    ax.set_xlabel("$Wavelenght [\\AA]$")                                                                            
    ax.set_ylabel("$Energy [10^{-17} erg/s/cm^2/\\AA]$")
    ax.axvline(x=spLine, color = 'g', ls ='--')
\end{lstlisting}%
\lthtmlfigureZ
\lthtmlcheckvsize\clearpage}


\renewcommand{\bibname}{References}
\appendix

\end{document}
