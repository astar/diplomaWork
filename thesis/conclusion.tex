\chapter{Conclusion}

% tohle zaver cele prace
The harvesting of large-scale astronomical data is challenging but
solvable problem. The technology of Virtual Observatory offers solid
background for data discovery and retrieval. The whole process can be
automated using UN*X like approach of small and single purpose scripts
or programs. The last stage of choosing the right characteristics and
data mining method is even more complex task requiring deep
understanding of the investigated phenomena and machine learning
theory and technology. The possible solution can be based on
cooperation between experts in scientific and computer science
field. Without such collaboration we are missing lots of
opportunities.

It is evident that dealing with spectroscopic data is much more
complicated but also more fruitful. One could extract many
characteristic features which fit to the actual problem. The FITS
standard is real godsend and makes work with spectra from different
sources possible. The results obtained from the data mining process
are reasonable. During the work it was also "discovered" how humans
are good at visual judgment: when thumbnails of resulting spectra were
created they provided much better understanding if something went
wrong then statistics and numbers. This is one example how machines
and humans could work together when we utilize ours and theirs natural
abilities.

There are many issues which could be done better, some of the
considered but not implemented subjects are discussed here:

\begin{itemize}
\item Spectral Characteristics 

  The spectrum was characterized with few, very simple parameters,
  which can be similar in different types of objects. We have
  discussed \footnote{Petr Skoda initiated rich and interesting email
    conversation with leading experts regarding this topic.} many
  advanced possibilities such as wavelets, eigenvalues etc. This could
  be subject of further investigation.

\item Continuum fit.

  The simple linear function is too rough to capture true continuum
  features. There is an interesting and effective algorithm discussed
  in the paper: Advanced fit technique for astrophysical spectra by
  S. Bukvi{\'c} et. al. from University of Belgrade
  \citep{bukvic2008advanced} which seems ideal for this purpose.

\item More data mining algorithms

Originally more advanced approaches such are Support Vector Machines
were considered.

\item Larger training sample.

  To obtain large enough meaningful training sample of confirmed Be
  stars was a real problem and many surveys were considered
  (e.g. IPHAS) but without success.

\item Using the time information.

  The whole process presented is static but the "Be phenomena" is
  dynamic in nature. Using light curves or time series of spectra
  could significantly improve the efficiency.
\item Unknown errors. 

  There are things we do not know we do not know. The overall process
  was very complicated and involved hundreds lines of code. Any
  mistake overlooked could affect the results. The absence of evidence
  is not evidence of absence.
\end{itemize}

Nevertheless even with these simple parameters some interesting results
were achieved: 

\begin{itemize}
\item Semi-automatic process from retrieving data through convolution
  up to data mining classification was implemented.
 
\item From the 178314 analyzed spectra 1110 were classified as Be
  candidates (but there are many imperfect spectra in SDSS). A sample
  of 46 objects is given in the Appendix 1.
\end{itemize}


