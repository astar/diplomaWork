\chapter{Be candidates}

\begin{figure}[!htbp]
  \begin{center}
    \leavevmode
    \ifpdf
    \includegraphics[scale = 1]{mapBeCandidates}
    \else
    \includegraphics[bb = 92 86 545 742, height=6in]{mapBeCandidates} 
    \fi
    \caption{Chapter structure}
    \label{FigStructure}
  \end{center}
\end{figure}


Astronomical objects used in this work to demonstrate some of the
discussed technologies and methods were Be stars. The goal was to
develop a process of finding new candidates in the available
data. Several approaches were considered and two of them are discussed
in the rest of this text. First one utilizes photometric properties of
Be stars, second uses spectra characteristics.


\section{Be stars}

The first example of Be star was reported by Padre Angelo Secchi in
his letter to the Astronomische Nachrichten in 1866.

Classical Be stars are non-supergiant B-type stars whose spectrum has
or had at some time, one or more Balmer lines in emission. The current
accepted explanation of this phenomena is circumstellar gaseous
component in the form of equatorial disk. Rapidly rotating central
star is important feature of these objects, which may be important
contributor of the circumstellar medium \cite{porter2003classical}.

The important characteristic used later in the work is the H$\alpha$
emission. The explanation of its origin is on following pictures.


    \begin{figure}[!htbp]
      \begin{center}
        \leavevmode
        \ifpdf
        \includegraphics[scale =.6]{beStarLine}
        \else
        \includegraphics[bb = 92 86 545 742, height=6in]{beStarLine}
        \fi
        \caption{Model of a typical Be star. Emission lines coming
          from an equatorial disk is added to the photo-spheric
          absorption spectrum. Central B star emits UV (Lyman
          continuum) and ionizes the disk, which in turn re-emits at
          high wavelength such as visible
          domain. \cite{hirata1984star}}
        \label{Figjhk_be_b}
      \end{center}
    \end{figure}


    \begin{figure}[!htbp]
      \begin{center}
        \leavevmode
        \ifpdf
        \includegraphics[scale =.6]{beStarLine2}
        \else
        \includegraphics[bb = 92 86 545 742, height=6in]{beStarLine2}
        \fi
        \caption{Example of spectra of Be stars based on view angle
          \cite{slettebak1988stars}}
        \label{Figjhk_be_b}
      \end{center}
    \end{figure}
 
\clearpage

There are still many open questions related to their rotation,
evolutionary status, presence and origin of the magnetic fields, mass
and angular momentum transfer and others, therefore the process for
automatic discoveries of Be phenomena in the digitalized surveys and
obtaining new candidates could help answer these questions.


\section{Photometric Data Mining}

% The question I have tried to answer in this chapter was: Is it
% possible to find Be stars candidates based on photometric properties
% only? To answer this question I needed training set of confirmed Be
% stars, set of non Be stars (spectral type B was considered) and some
% Data Mining algorithm to perform classification.

   \begin{figure}[!htbp]
      \begin{center}
        \leavevmode
        \ifpdf
        \includegraphics[scale =.5]{flowPhoto}
        \else
        \includegraphics[bb = 92 86 545 742, height=6in]{flowPhoto}
        \fi
        \caption{Schematic diagram of the photometric Data Mining
          process. The lists of confirmed Be stars consisted of
          Hipparcos IDs, this was correlated with Hipparcos catalog to
          obtain right ascension and declination of the objects and
          subsequently cross-matched with 2MASS catalog to get
          photometric data. The second set of B stars were acquired in
          similar manner but using SQL the condition was set to get B
          type stars different from the list of Be stars.}
        \label{FigFlowSpectra}
      \end{center}
    \end{figure}

 
\clearpage



Classification based on photometric properties is very attractive from
several points of view. There are much more available photometric then
spectral data and they are easier accessible. Because they are easier
to gain the disproportion between photometric and spectral data will
probably increase in the future as well. The distinction between Be
and other types of stars also should be theoretically possible since
the Be stars exhibits infrared excess correlated to the H$\alpha$
emission \cite{van1995halpha}.

\subsection{Data preprocessing}

I was provided with a list of confirmed Be stars from Academy of Science
Ondřejov. This list consist of 625 manually chosen objects. Data were
correlated with Hipparcos \cite{perryman1997hipparcos} catalog to
obtain RA, DEC and then with 2MASS\cite{2006AJ131.1163S} catalog to
obtain J,H,K Colors using method of multi-cone search in Virtual
Observatory. The second set was acquired from Hipparcos catalog using
following SQL query:

\begin{lstlisting}
  Select * 
  From maincat as m, hipva1 as h 
  Where  (m.HIP=h.HIP )  
  And h.SpType Like 'B%'
\end{lstlisting}

The result was cross-correlated with 2MASS catalog to obtain the same
colors as for the confirmed Be stars. Color digram of this two sets
are on the figure \ref{Figjhk_be_b}

    \begin{figure}[!htbp]
      \begin{center}
        \leavevmode
        \ifpdf
        \includegraphics[scale =.6]{jhk_be_b}
        \else
        \includegraphics[bb = 92 86 545 742, height=6in]{jhk_be_b}
        \fi
        \caption{Color diagram of confirmed Be stars Vs B stars}
        \label{Figjhk_be_b}
      \end{center}
    \end{figure}

    The uncertainties were computed for each object using propagation
    of error. These errors and depicted on the figure
    \ref{Figjhk_be_b_errors}. Although the uncertainties are
    significant certain trends are presented.

\begin{align*}
  \delta_{(j - h)} &= \sqrt{\left(\frac{\partial(j - h) }{\partial
        j}\right)^2\delta_j^2 + \left(\frac{\partial(j - h) }{\partial
        h}\right)^2\delta_h^2} \\
  \frac{\partial(j - h) }{\partial j } &= 1,\frac{\partial(j - h)
  }{\partial h } = -1 \\
  \delta_{(j - h)} &= \sqrt{\delta_j^2 + \delta_h^2}
\end{align*}


    \begin{figure}[!htbp]
      \begin{center}
        \leavevmode
        \ifpdf
        \includegraphics[scale =.6]{jhk_be_b_errors}
        \else
        \includegraphics[bb = 92 86 545 742, height=6in]{jhk_be_b_errors}
        \fi
        \caption{Color diagram of confirmed Be stars Vs B stars with errors}
        \label{Figjhk_be_b_errors}
      \end{center}
    \end{figure}

\clearpage


\subsection{Classification}
Data were transformed from original VOTable obtained from Virtual
Observatory tools to arff\footnote{Attribute-Relation File
  Format. Developed by the Machine Learning Project at the Department
  of Computer Science of The University of Waikato.}format used in
Weka Data Mining system. Algorithm C4.5 (J48) was used to perform
actual classification with following result:

\begin{lstlisting}
  Correctly Classified Instances         769               73.0989 %
  Incorrectly Classified Instances       283               26.9011 %
  Kappa statistic                          0.4496
  Mean absolute error                      0.3843
  Root mean squared error                  0.4383
  Relative absolute error                 79.4985 %
  Root relative squared error             89.1648 %
  Total Number of Instances             1052
\end{lstlisting}

As seen on the first row 73 \% from  1052 objects were classified
correctly. More details can be obtained from confusion matrix below.

\begin{lstlisting}
  B   Be   <-- classified as
 304 126 |   B
 157 465 |   Be
\end{lstlisting}

304 of B and 456 of Be stars were classified correctly but 126 of B
and 157 of Be stars were classified incorrectly. In virtue of these
results one should be sceptical if the distinction based only on
photometric properties is significant enough to find relevant new
candidates of Be stars. For this reason more sophisticated (and much
more complicated) approach using spectra analysis was tested.

\section{Spectral Data Mining}

   \begin{figure}[!htbp]
      \begin{center}
        \leavevmode
        \ifpdf
        \includegraphics[scale =.5]{flowSpectra}
        \else
        \includegraphics[bb = 92 86 545 742, height=6in]{flowSpectra}
        \fi
        \caption{Schematic diagram of the spectral Data Mining
          process. Using SSA protocol the spectra from Ondřejov server
          was acquired based on the list from photometric study. SSH
          Tunneling was necessary since Ondřejov spectra are top
          secret and therefore not available to the public. Convolution
          had to be performed to ensure compatibility with
          SDSS. Afterwards desired features were extracted
          automatically from the spectra after the continuum and
          H$\alpha$ line were fitted by appropriate functions. The
          same was done for spectra from SDSS except the convolution
          process.}
        \label{FigFlowSpectra}
      \end{center}
    \end{figure}

 
\clearpage

Spectra provide much wider scientific informations over photometric
properties. Spectral lines exhibits many distinguish features and
astronomers have long tradition of analysing their properties. On the
other side its much complicated to handle them because of different
characteristics (resolution, calibration, wavelength range, etc). This
is especially true for massive automated processing.  

\subsection{Testing Data}
As testing sample the project SEGUE of SDSS were selected. This
contains 178315 spectra in DR7. Following SQL query was used to
generate the list of URL links for individual FITS files. These files
were then downloaded to local sever using wget command.

\begin{lstlisting}
SELECT  objid,dbo.fGetUrlFitsSpectrum(s.specObjID)                                                           
INTO mydb.segue_1                                                                                     
FROM SpecPhotoAll s, platex p                                                                         
WHERE s.specObjID is not null                                                                         
AND s.plateid = p.plateid                                                                             
AND p.programname LIKE 'segue%'                                                                       
AND specClass = 1
\end{lstlisting}

\subsection{Training Data}
The spectra from Ondřejov Observatory were used as a training
sample. Files were downloaded using SSA protocol. The SSA server is
not publicly available, therefore SSH tunneling was used. Two scripts
for this process were created. First to construct the list of SSA
compliant addresses, the second to analyse acquired response in
VOTable format. Then the spectra were downloaded using wget
command. The function for constructing the links based on list of the
RA, DEC which were obtained from Hipparcos catalog using the
specification of IDs from Ondřejov's index.

\begin{lstlisting}
def createQuery(data):
    """ From raw data construct ra, dec """
    """ Convert to degrees """
    for line in data:
        ra = ac.AngularCoordinate(line[0:10]).degrees # convert ra to degrees
        dec = ac.AngularCoordinate(line[-13:-1]).degrees # convert dec to degrees
        ra = line[0]
        dec = line[1]
        ssaTemp = 'http://tvoserver/coude/coude.cgi?c=ssac&n=coude_ssa&REQUEST=queryData&POS=<ra>,<dec>&SIZE=1'
        ssaTemp = ssaTemp.replace('<ra>',"%0.3f" % ra)
        ssaTemp = ssaTemp.replace('<dec>',"%0.3f" % dec)
        ssa.append(ssaTemp)
    return ssa
\end{lstlisting}

The script generate the following output. The same process were used
later for obtaining th sample of non Be stars.

\begin{lstlisting}
http://tvoserver/coude/..._ssa&REQUEST=queryData&POS=83.113,-65.582&SIZE=60
http://tvoserver/coude/..._ssa&REQUEST=queryData&POS=162.537,148.333&SIZE=60
http://tvoserver/coude/..._ssa&REQUEST=queryData&POS=19.907,-73.502&SIZE=60
\end{lstlisting}


\subsubsection{Spectra Reduction}
Because spectra from SDSS and Ondřejov Observatory had different
resolution, reduction was needed. First the parameter CD1\_1 (Coordinate
increment per pixel) had to be obtained form FITS file.


\begin{lstlisting}
  In [1]: hdu = pf.open('sdss_test.fit')
  In [2]: hdu[0].header['CD1_1']
  Out[2]: 0.0001 # SDSS spectrum 
  Out[3]: 0.2567 # Onřejov spectrum
\end{lstlisting}

Spectra in SDSS are stored in logarithmic scale thus the value is
computed as $ 10^{CD1\_1} = 1.00$. The ratio is then
$CD1\_1_{SDSS}/CD1\_1_{OND} = 3.87$. Based on this computation 4
pixels of Ondřejov's spectra were reduced into one. There is the
critical part of the reduction program:

\begin{lstlisting}
 def convolution(f, g):
    """ Convolve two functions"""
    fg = np.convolve(g,f,'same')
    return fg
 def reduce(x,y,bin):
    """ Reduce bin pixel into 1"""
    size = x.size/bin
    l = 0
    xx = x[:x.size-1:bin]
    yy = list()
    for i in range(0,size):
        s = 0
        for j in range(0,bin):
            s = s + y[l]
            l+=1
        yy.append(s/bin)
    return xx, yy
\end{lstlisting}

Prior to binning pixels convolution with Gaussian function was
performed on the spectra. Convolution is defined:

\begin{equation}
  \label{eq:convolution}
 (f * g )(t) \stackrel{\mathrm{def}}{=}\ \int_{-\infty}^{\infty} f(\tau)\, g(t - \tau)\, d\tau
\end{equation}
 
Here it was used in it's discrete form

\begin{equation}
  \label{eq:discreteConvolution}
  (f * g)[n]\ \stackrel{\mathrm{def}}{=}\ \sum_{m=-\infty}^{\infty} f[m]\, g[n - m]
\end{equation}



The figure shows the result.


    \begin{figure}[!htbp]
      \begin{center}
        \leavevmode
        \ifpdf
        \includegraphics[scale =.8]{convolution}
        \else
        \includegraphics[bb = 92 86 545 742, height=6in]{convolution}
        \fi
        \caption{Reduction of Ondřejov's spectra of the Be star 4
          Hercules. The top figure shows Gaussian function used for
          convolution with the spectrum, followed by the original
          spectrum then there is a spectrum after convolution with the
          Gaussian function. The last is the final spectrum after
          reduction.}
        \label{FigReduction}
      \end{center}
    \end{figure}

\clearpage




\subsection{Spectra Lines Characteristics}
As parameters for Data Mining process characteristic values of
H$\alpha$ line were extracted from the spectra. Many possible
characteristics from fitting functions through Wavelets Coefficients
and Eigenvalues Values were discussed with experts. Three parameters
were finally selected. The hight and the width of the H$\alpha$
emission line and median absolute deviation as a characterization of
the noise level in the spectrum.


\subsubsection{Normalization}
Spectra from SDSS are normalized but the spectra from Ondřejov are
not. The spectra were divided by it's continuum fit function. This
process ensures the compatibility when comparing different
spectra. Function polyfit from numpy package was used to perform the
fit. The solution minimizes the squared error:

\begin{align}
  \frac{d}{dq} \sum_{i = 1}^n{(y_i - \operatorname{f}(x_i) )^2} = 0,
\end{align}

where $\operatorname{f}$ is in our case $\operatorname{f}(x) = q_1x +
q_0$.

\subsubsection{The hight of the H$\alpha$ line}
The maximum value in the region of $50\AA$ were extracted from the
spectrum.

\begin{lstlisting}
def getMax(x,y,line,range):
    """ Return maximum value of range in the spectrum"""
    xrange = x[(x < line + range) & (x > line - range)]
    yrange = y[(x < line + range) & (x > line - range)] - 1
    maximum = yrange.max()
    minimum = yrange.min()
    if abs(maximum)  > abs(minimum):
        extrem =  maximum 
    else:
        extrem = minimum 
    return xrange, extrem, sgn
\end{lstlisting}

\subsubsection{The noise level of the spectrum}
The noise in the spectrum contributes to the characteristics of the
spectral lines. As an estimator of the noise level the median
absolute deviation was used. It is defined as:

\begin{align}
  \operatorname{mad} = \operatorname{median}_{i}\left(\ \left| X_{i} -
      \operatorname{median}_{j} (X_{j}) \right|\ \right)
\end{align}

\subsubsection{The width of the H$\alpha$ line}
The Gaussian function was fitted to the spectral line. First the
robust estimators were computed and used as input parameters for
leastsq\footnote{"leastsqi"s a wrapper around MINPACK’s lmdif and
  lmder algorithms.} method from scipy.opt module, which minimize the
sum of squares.

\begin{align}
  x_0 & = \frac{\operatorname{median}(w_jx_j)}{\sum{w_i}}, \\
  S & = \frac{\operatorname{mad}(x_i - x_0)}{\sum{w_i}}.
\end{align}

\cite{launer1979robustness}

Part of the script implementing fitting the Gaussian function

\begin{lstlisting}
x0 = np.median(sum(w*x))/sum(w)
S = sum(w*mad((x - x0)))/sum(w)
params = np.array([1, maximum, x0, S], dtype=float)
fit, flag = opt.leastsq(residuals, params, args=(yrange, xrange))
gauss = model(xrange, fit) + 1

def model(t, coeffs):
    return coeffs[0] + coeffs[1] * np.exp( - ((t-coeffs[2])/coeffs[3])**2 )
def residuals(coeffs, y, t):
    return y - model(t, coeffs)
\end{lstlisting}

The final result is on the figure \ref{FigSpecChar1} and
\ref{FigSpecChar2}. The script was adjust to work with SDSS and
Ondřejov's spectra. The whole procedure was performed on all of the
cca 200 000 SDSS spectra and few dozens Ondřejov's spectra resulting
with the ASCII files with the characteristic values used later in Data
Mining process.

   \begin{figure}[!htbp]
      \begin{center}
        \leavevmode
        \ifpdf
        \includegraphics[scale =.8]{figSpecCharCyg60}
        \else
        \includegraphics[bb = 92 86 545 742, height=6in]{figSpecCharCyg60}
        \fi
        \caption{Normalized spectrum of Be star 60 Cyg. The top figure
          depicts the continuum fit. The bottom figure shows the
          region (width of the green line) used for extraction. The
          position of the line correspond to the maximum value in the
          region of $50\AA$. The Gaussian fit is in red. Although the
          fit is almost perfect, this approach fails to get
          characteristic "double peak" of the emission line. }
        \label{FigSpecChar}
      \end{center}
    \end{figure}

   \begin{figure}[!htbp]
      \begin{center}
        \leavevmode
        \ifpdf
        \includegraphics[scale =.8]{figSpecCharhd216057}
        \else
        \includegraphics[bb = 92 86 545 742, height=6in]{figSpecCharhd216057}
        \fi
        \caption{Normalized spectrum of Be star HR 8682. The top
          figure depicts the continuum fit. The bottom figure shows
          the region (width of the green line) used for
          extraction. The position of the line correspond to the
          maximum value in the region of $50\AA$. The Gaussian fit is
          in red.}
        \label{FigSpecChar}
      \end{center}
    \end{figure}


\clearpage

% The script was written to normalize the spectrum and extract the line
% characteristic value. This program also plots the results of the
% process as it is shown on previous picture. The function used to
% extract the line characteristic value is below.



\subsection{Data Mining}
Classification was performed using Weka software with algorithm J48
described in the chapter \ref{chap:dataMining}. Training set had 173
and testing set 178314 items. The excerpt from these files follows.

\begin{lstlisting}
@RELATION STAR-B-BE
@ATTRIBUTE name STRING
@ATTRIBUTE alpha NUMERIC
@ATTRIBUTE grp {be,o}
@DATA
10_cas,-0.822196556626,be
11_cyg,1.68689566629,be
\end{lstlisting}

\begin{lstlisting}
@RELATION STAR-B-BE
@ATTRIBUTE name STRING
@ATTRIBUTE alpha NUMERIC
@ATTRIBUTE grp {be,o}
@DATA	 
spSpec-53228-1884-001	-0.584628294569	 ?
spSpec-53228-1884-002	-0.877184482566	 ?
\end{lstlisting}

The attribute \textrm{grp} is known for the training set but unknown
for testing set. The classification process fills this information
based on decision tree created during learning phase. To automate the
process command line version of Weka software was used.

\begin{lstlisting}
  java -classpath weka.jar
  weka.classifiers.meta.FilteredClassifier -F
  weka.filters.unsupervised.attribute.RemoveType -W
  weka.classifiers.trees.J48 -t $1 -T $2 -p 1
\end{lstlisting}

% odsud prepsat
\subsection{Results}

% Because only one parameter (H$\alpha$) was used, the decision tree is
% very simple. If the value of the parameter is greater than $-0.464633$
% the object is considered to be a Be star. If the values in the range
% of $-0.676474$ and $-0.464633$ it is considered to be non Be star (no
% further restriction was assert on the \textrm{other} group). It imply
% that according to classifier the Be star is an object with extreme
% values in H$\alpha$ line. This outcome does not oppose our
% understanding of these kind of objects.

% pouzit jako ukazku jendoduchehho stromu
The overall fruitfulness of the classification process is almost
84\%. 10 folds cross-validation was used to compute the error
rate. 

% The training sample consists only of 173 objects. Also it is
% clear that used features are not sensitive enough to distinguess huge
% varieties and small differences between Be stars and other objects. On
% the other side the performance on the training sample is very
% promising

\begin{lstlisting}
  === Summary ===
Correctly Classified Instances         145               83.815  %
Incorrectly Classified Instances        28               16.185  %
Kappa statistic                          0.6529
Mean absolute error                      0.1849
Root mean squared error                  0.3652
Relative absolute error                 39.8819 %
Root relative squared error             75.8919 %
Total Number of Instances              173     
\end{lstlisting}

The classification tree is relatively complicated. But still we can
learn few things. It is using all of the parameters putted in so they
are choose correctly (if they were irrelevant classifier would not
used them). The most important parameter was \textrm{max} which
determines the hight of the line above the continuum. This was
expected as H$\alpha$ emission is dominated feature of Be stars. The
second important parameter was the noise of the spectrum expressed in
parameter \textrm{mad}. The less important (at least in this example)
was the width of the line. It need to be emphasized that the parameter
\textrm{max} does not really measure the hight, \textrm{mad} the noise
or \textrm{width} the width of the line but there are some simplified
(and certainly buggy) versions of real parameters. Though it can give
us some physical insight of the studied phenomenas. Decision trees are
therefore very powerful compared to black box approaches such are
Neural Networks where the classification process is beyond human
understanding.

\begin{lstlisting}
  J48 pruned tree
------------------
max <= -0.18843
|   max <= -0.324763: o (46.0/5.0)
|   max > -0.324763
|   |   max <= -0.255475
|   |   |   mad <= 0.004133: o (2.0)
|   |   |   mad > 0.004133: be (13.0/1.0)
|   |   max > -0.255475
|   |   |   mad <= 0.009862: o (10.0)
|   |   |   mad > 0.009862
|   |   |   |   width <= 7.621593: o (3.0/1.0)
|   |   |   |   width > 7.621593: be (2.0)
max > -0.18843
|   mad <= 0.030316
|   |   max <= -0.091726
|   |   |   width <= 5.286489
|   |   |   |   max <= -0.170022: be (2.0)
|   |   |   |   max > -0.170022: o (3.0)
|   |   |   width > 5.286489: be (9.0)
|   |   max > -0.091726: be (76.0)
|   mad > 0.030316
|   |   max <= 6.917615: o (4.0)
|   |   max > 6.917615: be (3.0)
\end{lstlisting}


\begin{lstlisting}
  === Confusion Matrix ===
 Be Others   <-- classified as
 95 15   | Be
 13 50   | Others
\end{lstlisting}

The Confusion Matrix evince that the classifier is more successful in
assigning Be stars (95/15) than in the case of others types stars
where 13/50 were associated with wrong class.



Spectra of some of the objects classified as Be stars are presented
here. These samples represent tiny fraction of complete result. The
program for generating web pages with thumbnails were created and full
result is available on the Wiki pages of this
project 

\url{http://physics.muni.cz/~vazny/wiki/index.php/Diploma_work}.

\begin{table}[ht]
%  \centering
  \small
     \begin{tabular}[ht]{c l c c c c c c c}
       \toprule 
     \# &SDSS name & RA & DEC & u  & g & r & i \\
   \midrule
   1&SDSS J035747.16-063850.7& 59.44 & -6.64& 19.83 &19.99& 19.73&19.86 \\ 
   2&SDSS J094325.89+520128.6& 145.86& 52.02& 16.57 &16.42& 16.55& 16.70\\ 
   3&SDSS J120729.12+003659.8& 181.87& 0.62 & 17.57 &15.28& 14.30& 13.96\\ 
   4&SDSS J120908.18+194035.8& 182.3 & 19.7 & 17.87 &16.26& 15.52& 15.19\\ 
   \bottomrule
   \end{tabular}
  \caption{Examples of the result}
  \label{tab:Result}
\end{table}


\begin{table}[ht]
%  \centering
  \small
     \begin{tabular}[ht]{c l}
       \toprule
     \# & link \\
   \midrule
   1& \url{http://cas.sdss.org/dr7/en/tools/explore/obj.asp?sid=583165493179842560} \\   
   2& \url{http://cas.sdss.org/dr7/en/tools/explore/obj.asp?sid=671267254834298880}\\ 
   3& \url{http://cas.sdss.org/dr7/en/tools/explore/obj.asp?sid=814259934286839808}\\ 
   4& \url{http://cas.sdss.org/dr7/en/tools/explore/obj.asp?sid=814541407275450368}\\ 
   \bottomrule
   \end{tabular}
  \caption{Links to objects on SDSS Skyserver}
  \label{tab:Links}
\end{table}





   \begin{figure}[!htbp]
      \begin{center}
        \leavevmode
        \ifpdf
        \includegraphics[scale =.6]{result1}
        \else
        \includegraphics[bb = 92 86 545 742, height=6in]{result1}
        \fi
        \caption{Spectrum of }
        
        \label{FigResult1}
      \end{center}
    \end{figure}

   \begin{figure}[!htbp]
      \begin{center}
        \leavevmode
        \ifpdf
        \includegraphics[scale =.6]{result2}
        \else
        \includegraphics[bb = 92 86 545 742, height=6in]{result2}
        \fi
        \caption{Spectrum of }
        \label{FigResult2}
      \end{center}
    \end{figure}

   \begin{figure}[!htbp]
      \begin{center}
        \leavevmode
        \ifpdf
        \includegraphics[scale =.6]{result3}
        \else
        \includegraphics[bb = 92 86 545 742, height=6in]{result1}
        \fi
        \caption{Spectrum of }
        \label{FigResult3}
      \end{center}
    \end{figure}

   \begin{figure}[!htbp]
      \begin{center}
        \leavevmode
        \ifpdf
        \includegraphics[scale =.6]{result3}
        \else
        \includegraphics[bb = 92 86 545 742, height=6in]{result1}
        \fi
        \caption{Spectrum of }
        \label{FigResult3}
      \end{center}
    \end{figure}


% Samples of Be Stars           
 
    For comparison there are spectra of know Be stars. It is clear
    that the profile of the H$\alpha$ line is complex and just one
    parameter cannot possibly express it's characteristic. More
    advanced description such as Wavelets coefficients or theoretical
    models of the line is needed if we want to create reliable
    process for identifying Be stars.

   \begin{figure}[!htbp]
      \begin{center}
        \leavevmode
        \ifpdf
        \includegraphics[scale =.6]{be1_4_her}
        \else
        \includegraphics[bb = 92 86 545 742, height=6in]{be1_4_her}
        \fi
        \caption{Spectrum of 4 Her. Be star. Spectral Type B9pe.  }
        \label{FigBe1}
      \end{center}
    \end{figure}

   \begin{figure}[!htbp]
      \begin{center}
        \leavevmode
        \ifpdf
        \includegraphics[scale =.6]{be4_bet_cyg_b}
        \else
        \includegraphics[bb = 92 86 545 742, height=6in]{be4_bet_cyg_b}
        \fi
        \caption{Spectrum of HR 7418 (Albireo B). A fast-rotating Be
          star, with an equatorial rotational velocity of at least 250
          kilometers per second. Its surface temperature has been
          spectroscopically estimated to be about 13.200 K. Spectral
          Type B8Ve. }
        \label{FigBe4}
      \end{center}
    \end{figure}

   \begin{figure}[!htbp]
      \begin{center}
        \leavevmode
        \ifpdf
        \includegraphics[scale =.6]{be3_6_cep}
        \else
        \includegraphics[bb = 92 86 545 742, height=6in]{be3_6_cep}
        \fi
        \caption{Spectrum of 6 Cepheus. Be star. Spectral Type B3IVe.}
        \label{FigBe3}
      \end{center}
    \end{figure}



\clearpage

\subsection{Experiment}

One could be interested what would happened if we have choose
different parameters, used other algorithm, different training set
etc. These are perfectly valid questions and it is actually the purpose
and essence of Data Mining and computers in general: perform similar
task over and over again. With some automation in mind such
experiments are easy to do. Here is one of the test I have performed.

Natural idea one could have is using just emission line hight and
write a program to check the spectra for condition \textrm{if emission
  > threshold}. Then we do not need "expensive" Data Mining algorithms to
get some interesting results\footnote{This was actually done in the early
stage of this project.}. Lets try to use Classification using just
emission line parameter. Here is consequent tree.


\begin{lstlisting}
  J48 pruned tree
------------------
alpha <= -0.464633
|   alpha <= -0.676474: be (45.0/18.0)
|   alpha > -0.676474: o (46.0/5.0)
alpha > -0.464633: be (92.0/16.0)
\end{lstlisting}

We can see that it is not that simple. In the training set there are
some Be stars with extreme hight in the negative direction (absorption
line). We have some statistics from classifier. In this example
the effectiveness were $77.6$\%. Also there is Confusion Matrix 

\begin{lstlisting}
  === Confusion Matrix ===
 Be     O   <-- classified as
 102    6 |   Be
  35   40 |   Others
\end{lstlisting}

which indicates how the classifier will fail to perform in individual
cases. Here it shows that it is almost exact when the object is Be
star but it confuses others type of stars with Be stars.

This experiment proves that using Data Mining has a sense even in
simple cases and can provide non-trivial insight on tested data.


% Confusion Matrix indicates that the classifier is much better in
% predicting Be stars (102/6 correct assignments) then in predicting
% results in the \textrm{others} group where there are just 40/35 right
% assignments.



\subsection{Conclusion}

It is evident that dealing with spectroscopic data is much more
complicated but also more fruitful. One could extract many
characteristic features which fit to the actual problem. The FITs
standard is real godsend and makes work with spectra from different
sources possible. The results obtained from the Data Miming process
are reasonable. During the work it was also "discovered" how humans
are good at visual judgment: when thumbnails of result spectra were
created it provided much better understanding if something went wrong
then statistics and numbers. This is one example how machines and
humans could work together when we utilize ours and theirs natural
abilities.

There are many aspect which could be done better, some of the
considered but not implemented subjects are discussed here.

\begin{itemize}
\item Spectral Characteristics 

  The spectrum was characterized with few, very simple parameters,
  which can be similar in different types of objects. We have
  discussed \footnote{Petr Skoda initiated rich and interesting email
    conversation with leading experts regarding this topic.} many
  advanced possibilities such are wavelets eigen values etc. This
  could be subject of further investigation.
\item Continuum fit

  The simple linear function is too rough to capture true continuum
  features. There is an interesting and effective algorithm discussed
  in the paper: Advanced fit technique for astrophysical spectra by
  S. Bukvi{\'c} et. al. from University of Belgrade
  \cite{bukvic2008advanced} which seems ideal for this purpose.

\item More Data Mining algorithms

Originally more advanced approaches such are Support Vector Machines
were considered.
\item Larger training sample

  To obtain large enough meaningful training sample of confirmed Be
  stars was a real problem and  many surveys were consider
  (e.g. IPHAS)  but without success.

\item Usage of Light curves 

  The whole process static but the "Be phenomena" is dynamic in
  nature. Using light curves could significantly improve the
  efficiency.
\item Unknown errors There are things we do not know we don't

  know. The overall process were very complicated and involved
  hundreds lines of codes. Any overlooked mistake could effect the
  results. The absence of evidence is not evidence of absence.
\end{itemize}



