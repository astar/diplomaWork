
% Thesis Abstract -----------------------------------------------------


%\begin{abstractslong}    %uncommenting this line, gives a different abstract heading
\begin{abstracts}        %this creates the heading for the abstract page


  Modern astrophysics and natural sciences in general become
  extensively penetrated by Computer Science. Petabyte scale
  databases, GRID Computing and Data Mining become the routine part of
  scientific work. Skills related to information technologies are now
  essential. The new concept of data infrastructure named Virtual
  Observatory naturally emerged from digitized surveys. Based on
  proven standards it offers an ideal basis for dealing with distributed
  heterogeneous data. This thesis is a case study of using Virtual
  Observatory and Data Mining technologies to proceed automatics
  classification of Be stars. Photometric and spectra classification
  were done on a large scale sample of almost 200 000 spectra from SDSS
  Segue survey.

  Many byproduct originated during the work on the thesis. Wiki pages,
  Virtual Observatory and Data Mining documentation and about dozen of
  programs for data manipulation, spectra fitting and result
  publishing. Everything is given free to the public on the web of the
  project.


  \url{http://physics.muni.cz/~vazny/wiki/index.php/Diploma_work}

\end{abstracts}
\cfoot{\thepage}

%\end{abstractlongs}


% ----------------------------------------------------------------------


%%% Local Variables: 
%%% mode: latex
%%% TeX-master: "../thesis"
%%% End: 
