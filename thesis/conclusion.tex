\chapter{Conclusion}

% tohle zaver cele prace
The harvesting of large-scale astronomical data is challenging but
solvable problem. The technology of Virtual Observatory offers solid
background for data discovery and retrieval. The whole process can be
automated using UN*X like approach of small and single purpose scripts
or programs. The last stage of choosing the right characteristics and
Data Mining method is even more complex task requiring deep
understanding of the researched phenomenas and Machine Learning theory
and technology. The possible solution can be based on cooperation
between experts in scientific and computer science field. Without such
collaboration we are missing lots of opportunities.

It is evident that dealing with spectroscopic data is much more
complicated but also more fruitful. One could extract many
characteristic features which fit to the actual problem. The FITs
standard is real godsend and makes work with spectra from different
sources possible. The results obtained from the Data Miming process
are reasonable. During the work it was also "discovered" how humans
are good at visual judgment: when thumbnails of result spectra were
created it provided much better understanding if something went wrong
then statistics and numbers. This is one example how machines and
humans could work together when we utilize ours and theirs natural
abilities.

There are many aspect which could be done better, some of the
considered but not implemented subjects are discussed here.

\begin{itemize}
\item Spectral Characteristics 

  The spectrum was characterized with few, very simple parameters,
  which can be similar in different types of objects. We have
  discussed \footnote{Petr Skoda initiated rich and interesting email
    conversation with leading experts regarding this topic.} many
  advanced possibilities such are wavelets eigen values etc. This
  could be subject of further investigation.
\item Continuum fit

  The simple linear function is too rough to capture true continuum
  features. There is an interesting and effective algorithm discussed
  in the paper: Advanced fit technique for astrophysical spectra by
  S. Bukvi{\'c} et. al. from University of Belgrade
  \citep{bukvic2008advanced} which seems ideal for this purpose.

\item More Data Mining algorithms

Originally more advanced approaches such are Support Vector Machines
were considered.
\item Larger training sample

  To obtain large enough meaningful training sample of confirmed Be
  stars was a real problem and  many surveys were consider
  (e.g. IPHAS)  but without success.

\item Usage of Light curves 

  The whole process static but the "Be phenomena" is dynamic in
  nature. Using light curves could significantly improve the
  efficiency.
\item Unknown errors 

  There are things we do not know we don't know. The overall process
  were very complicated and involved hundreds lines of codes. Any
  overlooked mistake could effect the results. The absence of evidence
  is not evidence of absence.
\end{itemize}





