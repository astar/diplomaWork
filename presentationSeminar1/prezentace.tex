% $Header: /cvsroot/latex-beamer/latex-beamer/solutions/generic-talks/generic-ornate-15min-45min.en.tex,v 1.5 2007/01/28 20:48:23 tantau Exp $

\documentclass[fleqn]{beamer}

% This file is a solution template for:

% - Giving a talk on some subject.
% - The talk is between 15min and 45min long.
% - Style is ornate.



% Copyright 2004 by Till Tantau <tantau@users.sourceforge.net>.
%
% In principle, this file can be redistributed and/or modified under
% the terms of the GNU Public License, version 2.
%
% However, this file is supposed to be a template to be modified
% for your own needs. For this reason, if you use this file as a
% template and not specifically distribute it as part of a another
% package/program, I grant the extra permission to freely copy and
% modify this file as you see fit and even to delete this copyright
% notice. 


\mode<presentation>
% {
%   \usetheme{Warsaw}
%   % or ...

%   \setbeamercovered{transparent}
%   % or whatever (possibly just delete it)
% }

\usepackage{beamerthemeshadow}
\usepackage[english,czech]{babel}
\usepackage{bibentry}
%\usepackage{natbib}
%\selectbiblanguage{czech}
\usepackage{url}
\usepackage{hyperref}
\usepackage{lmodern}
\def\uv#1{\glqq#1\grqq}
\usepackage{graphicx}
\usepackage{listings}
% or whatever
\usepackage{amsmath}
\usepackage{xspace}
\usepackage{natbib}
\usepackage{float}
%\usepackage{wrapfig}
%\usepackage{sidecap}
%\usepackage{txfonts}            
\usepackage{color}



\usepackage[utf8]{inputenc}
% or whatever

%\usepackage{times}
\usepackage[T1]{fontenc}
% Or whatever. Note that the encoding and the font should match. If T1
% does not look nice, try deleting the line with the fontenc.


\title[Dobývání znalostí z astronomických dat] % (optional, use only with long paper titles)
{Studium astronomických objektů s použitím technologií Virtuální observatoře} 

%\subtitle
%{Nástroje pro získávání a zpracování dat} % (optional)

\author[Jaroslav Vážný] % (optional, use only with lots of authors)
{Jaroslav Vážný }
% - Use the \inst{?} command only if the authors have different
%   affiliation.

\institute[Universities of Somewhere and Elsewhere] % (optional, but mostly needed)
{

    Masarykova univerzita

}
% - Use the \inst command only if there are several affiliations.
% - Keep it simple, no one is interested in your street address.

%\date[Short Occasion] % (optional)
%{Date / Occasion}

\subject{Talks}
% This is only inserted into the PDF information catalog. Can be left
% out. 



% If you have a file called "university-logo-filename.xxx", where xxx
% is a graphic format that can be processed by latex or pdflatex,
% resp., then you can add a logo as follows:

\pgfdeclareimage[height=0.5cm]{university-logo}{sci-logo}
\logo{\pgfuseimage{university-logo}}



% Delete this, if you do not want the table of contents to pop up at
% the beginning of each subsection:
\AtBeginSubsection[]
{
  \begin{frame}<beamer>{Outline}
    \tableofcontents[currentsection,currentsubsection]
  \end{frame}
}


% If you wish to uncover everything in a step-wise fashion, uncomment
% the following command: 

%\beamerdefaultoverlayspecification{<+->}


\begin{document}
% vylepseny listing z http://texnik.de/TeXnik/listings/listing0.pdf
 \definecolor{hellgelb}{rgb}{1,1,0.8}
 \definecolor{colKeys}{rgb}{0,0,1}
 \definecolor{colIdentifier}{rgb}{0,0,0}
 \definecolor{colComments}{rgb}{1,0,0}
 \definecolor{colString}{rgb}{0,0.5,0}
 \lstset{%
   language={SQL},%
    morekeywords={AND,ASC,avg,CHECK,COMMIT,count,DECODE,DESC,DISTINCT,%
                 GROUP,IN,LIKE,NUMBER,ROLLBACK,SUBSTR,sum,VARCHAR2}%
 }

 \lstset{%
     float=hbp,%
     basicstyle=\ttfamily\small, %
     identifierstyle=\color{colIdentifier}, %
     keywordstyle=\color{colKeys}, %
     stringstyle=\color{colString}, %
     commentstyle=\color{colComments}, %
     columns=flexible, %
     tabsize=4, %
     frame=single, %
     extendedchars=true, %
     showspaces=false, %
     showstringspaces=false, %
   numbers=left, %
   numberstyle=\tiny, %
   breaklines=true, %
   backgroundcolor=\color{hellgelb}, %
   breakautoindent=true, %
   captionpos=b%
 }

 

%\begin{frame}
%  \begin{center}
%    \frametitle{Vítejte ...}
%    \small{Anybody who is not shocked by this subject has failed to understand it.}\cite{skala2005ukm} 
%  \end{center}
%\end{frame}



\begin{frame}
  \titlepage
\end{frame}


%\begin{frame}
%  \tableofcontents
%\end{frame}

\begin{section}{Zadání}
\begin{frame}\frametitle{Zadání}
    Současné astronomické archivy obsahují nejen Petabyty informací o
    stovkách milionů objektů v celém rozsahu elektromagnetického
    spektra, ale navíc další data přibývající exponenciálním tempem
    jsou ukládána na desítkách serverů jednotlivých projektů
    roztroušených po celé zeměkouli.  Efektivní analýza dat v
    takovémto prostředí pak vyžaduje \textcolor{green}{zcela nový
      systémový přístup} a odlišné nástroje používající technologie a
    infrastrukturu astronomické Virtuální observatoře. Jejich správné
    využití spolu s metodami dobývání znalostí z dat (data mining) pak
    umožňuje získat principiálně nové informace o fyzikální podstatě
    astronomických objektů na základě jejich multispektrálních
    charakteristik či statistické výjimečnosti.  \textcolor{red}{Cílem
      práce je studium charakteristik vybrané třídy hvězd či
      extragalaktických objektů s použitím Virtuální observatoře.}
\end{frame}
\end{section}

\begin{section}{Motivace}
\begin{frame}\frametitle{Záplava dat (datová krize) v astrofyzice}
\begin{itemize}

\item  Tycho Brahe $500\:\rm kB$ 
\item  Sloan Digital Sky Survey: $3\:\rm TB$
\item  LHC (Atlas, CMS, LHCb, ALICE) $15\:\rm PB/rok$
\item  Large Synoptic Survey Telescope: $150\:\rm PB$ ($30\:\rm PB/noc$)

\end{itemize}

motto: Dat je moc a neumíme s nimi nakládat. I kdybychom to uměli,
nevíme co s nimi.

\end{frame}


\begin{frame}\frametitle{Počítače používáme špatně}

\begin{itemize}
\item  Co je špatně
    \begin{itemize}
    \item  Umělá inteligence (Podvod;-), opakování, lokální přístup    
    \end{itemize}
\item  Co je správně
  \begin{itemize}
  \item  Fraktály, SETI, sítě
  \end{itemize}
\end{itemize}

motto: Problém není v technologiích, ale v přístupu
\end{frame}
\end{section}

\begin{section}{Virtuální observatoř}
\begin{frame}\frametitle{Tim Berners-Lee www a raw data}

\begin{itemize}
\item  Skrytý potenciál a frustrace
\item  Raw data: Revoluce?
\item  Problém: Database hugging
\end{itemize}

\url{http://video.ted.com/talks/podcast/TimBerners-Lee_2009_480.mp4}

\end{frame}

\begin{frame}\frametitle{Co je to Virtuální observatoř}

\begin{itemize}
\item Infrastruktura pro přístup k datům
\item Sada protokolů (VOTable, FITS, SAMP)
\item Programy (Topcat, Aladin, VODesktop)
\end{itemize}
\end{frame}
\end{section}

\begin{section}{Dobývání/rýžování znalostí}

\begin{frame}\frametitle{Co je to data mining}

    Třídění a hledaní souvislostí.

    \begin{itemize}
    \item  Klasifikace

      \begin{itemize}
        \item  Na základě kontrolní skupiny roztřídíme data
      \end{itemize}

    \item  Shluková analýza      
      
      \begin{itemize}
      \item  Pomocí metriky najdeme vzdálenosti mezi objekty        
      \end{itemize}

    \end{itemize}
\end{frame}
\end{section}

\begin{section}{Stav}
\begin{frame}\frametitle{Jak jsem na tom?}
  
  \begin{itemize}
  \item Technologie

    \begin{itemize}
    \item  Virtuální observatoř
    \item  Astrogrid
    \item  Python
    \item  Weka
    \item Objekty
    \end{itemize}

  \begin{itemize}
  \item  Be Hvězdy
  \item  Blazary
  \end{itemize}


  
  \end{itemize}

\end{frame}

\begin{frame}
  \begin{center}
 \huge Děkuji za pozornost!    

  \end{center}

\end{frame}

\end{section}



\end{document}


